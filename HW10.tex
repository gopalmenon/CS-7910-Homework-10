%%%%%%%%%%%%%%%%%%%%%%%%%%%%%%%%%%%%%%%%%
% Short Sectioned Assignment
% LaTeX Template
% Version 1.0 (5/5/12)
%
% This template has been downloaded from:
% http://www.LaTeXTemplates.com
%
% Original author:
% Frits Wenneker (http://www.howtotex.com)
%
% License:
% CC BY-NC-SA 3.0 (http://creativecommons.org/licenses/by-nc-sa/3.0/)
%
%%%%%%%%%%%%%%%%%%%%%%%%%%%%%%%%%%%%%%%%%

%----------------------------------------------------------------------------------------
%	PACKAGES AND OTHER DOCUMENT CONFIGURATIONS
%----------------------------------------------------------------------------------------

\documentclass[paper=a4, fontsize=11pt]{scrartcl} % A4 paper and 11pt font size

\usepackage[T1]{fontenc} % Use 8-bit encoding that has 256 glyphs
\usepackage{fourier} % Use the Adobe Utopia font for the document - comment this line to return to the LaTeX default
\usepackage[english]{babel} % English language/hyphenation
\usepackage{amsmath,amsfonts,amsthm} % Math packages

\usepackage{sectsty} % Allows customizing section commands
\usepackage[top=5em]{geometry}
\allsectionsfont{\centering \normalfont\scshape} % Make all sections centered, the default font and small caps

\usepackage{fancyhdr} % Custom headers and footers
\pagestyle{fancyplain} % Makes all pages in the document conform to the custom headers and footers
\fancyhead{} % No page header - if you want one, create it in the same way as the footers below
\fancyfoot[L]{} % Empty left footer
\fancyfoot[C]{} % Empty center footer
\fancyfoot[R]{\thepage} % Page numbering for right footer
\renewcommand{\headrulewidth}{0pt} % Remove header underlines
\renewcommand{\footrulewidth}{0pt} % Remove footer underlines
\setlength{\headheight}{5pt} % Customize the height of the header

\numberwithin{figure}{section} % Number figures within sections (i.e. 1.1, 1.2, 2.1, 2.2 instead of 1, 2, 3, 4)
\numberwithin{table}{section} % Number tables within sections (i.e. 1.1, 1.2, 2.1, 2.2 instead of 1, 2, 3, 4)

\setlength\parindent{0pt} % Removes all indentation from paragraphs - comment this line for an assignment with lots of text

\usepackage{mathtools}
\usepackage{amssymb}
\usepackage{gensymb}
\usepackage{chngcntr}
\usepackage{csquotes}
\usepackage{flexisym}
\usepackage{algorithm,algpseudocode}
\usepackage{tikz}

\usepackage{verbatim}
\usetikzlibrary{arrows,shapes}

\newcommand\Mycomb[2][n]{\prescript{#1\mkern-0.5mu}{}C_{#2}}

\counterwithout{figure}{section}
%----------------------------------------------------------------------------------------
%	TITLE SECTION
%----------------------------------------------------------------------------------------

\newcommand{\horrule}[1]{\rule{\linewidth}{#1}} % Create horizontal rule command with 1 argument of height

\title{	
\normalfont \normalsize 
\textsc{Utah State University, Computer Science Department} \\ [25pt] % Your university, school and/or department name(s)
\horrule{0.5pt} \\[0.4cm] % Thin top horizontal rule
\huge CS 7910 Computational Complexity\\Assignment 10\\ % The assignment title
\horrule{2pt} \\[0.5cm] % Thick bottom horizontal rule
}

\author{Gopal Menon} % Your name

\date{\normalsize\today} % Today's date or a custom date

\begin{document}

\maketitle % Print the title

\begin{enumerate}
\item \textbf{(20 points)} In this exercise, we design an approximation algorithm for the dominating set
problem. We have proved in class that the dominating set problem is NP-Complete. Here we consider its optimization problem.

Given an undirected graph $G$ of $n$ vertices, a subset $S$ of vertices of $G$ is a dominating set if
each vertex $v$ of $G$ is either in $S$ or connects to a vertex of $S$ by an edge. The problem is to
find a dominating set of $G$ of minimum size.

Design a polynomial-time approximation algorithm for the problem with approximation ratio
$O(log n)$. In other words, if $OPT$ is the size of the optimal dominating set and $C$ is the size
of the dominating set found by your algorithm, then it should hold that $C \leq O(log n) \cdot OP T$,
which is equivalent to $C = O(OPT \cdot log n)$ by the definition of the big-O notation.

\item In this exercise, we consider a \enquote{dual} problem of the load balancing problem.

Suppose there are $m$ machines and $n$ jobs such that each job $i$ has a processing time $t_i$.
Consider a job assignment that assigns each job to one of these machines. For each machine
$j$, let $T_j$ denote the total sum of the processing time of all jobs assigned to machine $j$, and
we call $T_j$ the workload of machine $j$. We call the value $min_{1 \leq j \leq m} T_j$ the \textit{minimum workload} of all machines of the assignment.

The \textit{dual load balancing problem} is to compute a job assignment that \textit{maximizes} the minimum
workload of all machines.

\textbf{Remark.} Recall that the load balancing problem is to find a job assignment that minimizes
the maximum workload of all machines. Therefore, the two problem are \enquote{dual} to each other.
\begin{enumerate}
\item \textbf{(5 points)} The dual load balancing problem defined above is an optimization problem.
What is the decision version of this problem?
\item \textbf{(10 points)} Prove that the decision problem is NP-Complete.
\item \textbf{(15 points)} Let $A$ be the sum of the processing time of all jobs, i.e., $A = \sum\limits_{i=1}^n t_i$. We assume that $t_i \leq \frac{A}{2m}$ for each job $i$ (intuitively, each $t_i$ is not \enquote{too big}). Under this assumption, design a polynomial-time approximation algorithm for the problem with
approximation ratio 2. In other words, if $OPT$ is the minimum workload in an optimal
solution and $C$ is the minimum workload in your solution, then it holds that $C \geq \frac{1}{2} \cdot OPT$.
\end{enumerate}
\end{enumerate}

%----------------------------------------------------------------------------------------

\end{document}