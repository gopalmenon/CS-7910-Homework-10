%%%%%%%%%%%%%%%%%%%%%%%%%%%%%%%%%%%%%%%%%
% Short Sectioned Assignment
% LaTeX Template
% Version 1.0 (5/5/12)
%
% This template has been downloaded from:
% http://www.LaTeXTemplates.com
%
% Original author:
% Frits Wenneker (http://www.howtotex.com)
%
% License:
% CC BY-NC-SA 3.0 (http://creativecommons.org/licenses/by-nc-sa/3.0/)
%
%%%%%%%%%%%%%%%%%%%%%%%%%%%%%%%%%%%%%%%%%

%----------------------------------------------------------------------------------------
%	PACKAGES AND OTHER DOCUMENT CONFIGURATIONS
%----------------------------------------------------------------------------------------

\documentclass[paper=a4, fontsize=11pt]{scrartcl} % A4 paper and 11pt font size

\usepackage[T1]{fontenc} % Use 8-bit encoding that has 256 glyphs
\usepackage{fourier} % Use the Adobe Utopia font for the document - comment this line to return to the LaTeX default
\usepackage[english]{babel} % English language/hyphenation
\usepackage{amsmath,amsfonts,amsthm} % Math packages

\usepackage{sectsty} % Allows customizing section commands
\usepackage[top=5em]{geometry}
\allsectionsfont{\centering \normalfont\scshape} % Make all sections centered, the default font and small caps

\usepackage{fancyhdr} % Custom headers and footers
\pagestyle{fancyplain} % Makes all pages in the document conform to the custom headers and footers
\fancyhead{} % No page header - if you want one, create it in the same way as the footers below
\fancyfoot[L]{} % Empty left footer
\fancyfoot[C]{} % Empty center footer
\fancyfoot[R]{\thepage} % Page numbering for right footer
\renewcommand{\headrulewidth}{0pt} % Remove header underlines
\renewcommand{\footrulewidth}{0pt} % Remove footer underlines
\setlength{\headheight}{5pt} % Customize the height of the header

\numberwithin{figure}{section} % Number figures within sections (i.e. 1.1, 1.2, 2.1, 2.2 instead of 1, 2, 3, 4)
\numberwithin{table}{section} % Number tables within sections (i.e. 1.1, 1.2, 2.1, 2.2 instead of 1, 2, 3, 4)

\setlength\parindent{0pt} % Removes all indentation from paragraphs - comment this line for an assignment with lots of text

\usepackage{mathtools}
\usepackage{amssymb}
\usepackage{gensymb}
\usepackage{chngcntr}
\usepackage{csquotes}
\usepackage{flexisym}
\usepackage{algorithm,algpseudocode}
\usepackage{tikz}

\usepackage{verbatim}
\usetikzlibrary{arrows,shapes}

\newcommand\Mycomb[2][n]{\prescript{#1\mkern-0.5mu}{}C_{#2}}

\counterwithout{figure}{section}
%----------------------------------------------------------------------------------------
%	TITLE SECTION
%----------------------------------------------------------------------------------------

\newcommand{\horrule}[1]{\rule{\linewidth}{#1}} % Create horizontal rule command with 1 argument of height

\title{	
\normalfont \normalsize 
\textsc{Utah State University, Computer Science Department} \\ [25pt] % Your university, school and/or department name(s)
\horrule{0.5pt} \\[0.4cm] % Thin top horizontal rule
\huge CS 7910 Computational Complexity\\Assignment 10\\ % The assignment title
\horrule{2pt} \\[0.5cm] % Thick bottom horizontal rule
}

\author{Gopal Menon} % Your name

\date{\normalsize\today} % Today's date or a custom date

\begin{document}

\maketitle % Print the title

\begin{enumerate}
\item \textbf{(20 points)} In this exercise, we design an approximation algorithm for the dominating set
problem. We have proved in class that the dominating set problem is NP-Complete. Here we consider its optimization problem.

Given an undirected graph $G$ of $n$ vertices, a subset $S$ of vertices of $G$ is a dominating set if
each vertex $v$ of $G$ is either in $S$ or connects to a vertex of $S$ by an edge. The problem is to
find a dominating set of $G$ of minimum size.

Design a polynomial-time approximation algorithm for the problem with approximation ratio
$O(log n)$. In other words, if $OPT$ is the size of the optimal dominating set and $C$ is the size
of the dominating set found by your algorithm, then it should hold that $C \leq O(log n) \cdot OP T$,
which is equivalent to $C = O(OPT \cdot log n)$ by the definition of the big-O notation.\\

Consider the following approximate algorithm for finding a dominating set, that takes a graph $G$ having a set of $V$ vertices and $E$ edges as an input parameter

\begin{minipage}{\linewidth}
  \begin{algorithm}[H]
    \caption{Greedy Dominating Set Approximation Algorithm}\label{DomSetApproxAlg}
    \begin{algorithmic}[1]
      \Procedure{Greedy Dominating Set Approximation Algorithm}{$G$}
        \State Create remaining vertices set $R = V$
        \State Create an empty set of vertices $S = \phi$
        \While {$R \neq \phi$}
          \State Select vertex $v$ from $R$ such that set $S_i$ consisting of vertex $v$    
          \State and all vertices connected to $v$ by an edge maximizes $S_i \cap R$
          \State $R = R - S_i$
          \State $S= S \cup v$
        \EndWhile
        \State return $S$
      \EndProcedure
    \end{algorithmic}
  \end{algorithm}
\end{minipage}\\

For every element $x \in S_i \cap R$ in algorithm \ref{DomSetApproxAlg}, we can associate a cost $C_x = \frac{1}{S_i \cap R}$. If we add up the cost for every element in set $S_i$ we will get a total of $1$. The size of the dominating set found by this algorithm will be the number of vertices in set $S$. Since the dominating set will cover all vertices either by including them in $S$ or by having an edge to it from a vertex in set $S$, the number of vertices $C$ in the dominating set will be given by
\begin{align*} 
C = \sum \limits_{x \in V} C_x
\end{align*}
 We need to show that this algorithm will give us a dominating set of size 
\begin{align*} 
C \leq O(log n) \cdot OPT 
\end{align*}
where $OPT$ is the number of vertices in the optimal dominating set.

We will start by proving the following lemma for any subset of vertices $S_k$
\begin{align} \label{lemma}
\sum \limits_{x \in S_k} C_x \leq H(d_k) 
\end{align}
where $H$ is the harmonic function defined as
\begin{align} \label{HarmonicFunction}
H(n) = 1 + \frac{1}{2} + \frac{1}{3}+ \frac{1}{4} + ... + \frac{1}{n} = \theta (log n)
\end{align}and 
\begin{align*} 
d_k = \left | S_k \right | 
\end{align*}

Let 
\begin{align*} 
S_k = \{x_1, x_2, ..., x_{d_k}\} 
\end{align*}
be the set $S_k$ with its elements ordered by the time they are first removed from the set of remaining vertices $R$. Consider an element $x_j$ of $S_k$. Suppose  $x_j$ is first removed in an iteration when we remove $S_i$ from $R$. $x_j \in S_i \cap R$ at the beginning of the iteration. And $x_j, x_{j+1}, ..., x_{d_k}$ have not been removed from $R$ yet. Since the set $S_i$ may or may not be the same as $S_k$, 
\begin{align*} 
\left | S_i \cap R \right | \geq \left | S_k \cap R \right | \geq d_k - j + 1
\end{align*}
We can see that
\begin{align*} 
C_{x_j} = \frac{1}{S_i \cap R} \leq \frac{1}{d_k - j + 1} 
\end{align*} for any $1 \leq j \leq d_k$.\\
So
\begin{align} \label{IndividualCosts}
C_{x_1} \leq \frac{1}{d_k}, C_{x_2} \leq \frac{1}{d_k -1},..., C_{x_{d_k -1}} \leq \frac{1}{2}, C_{x_{d_k}} \leq 1
\end{align}
Adding everything in \ref{IndividualCosts}, we get
\begin{align} \label{HarmonicFnProof}
\sum \limits_{x \in S_k} C_x \leq 1 + \frac{1}{2} + \frac{1}{3}+ \frac{1}{4} + ... + \frac{1}{d{k-1}} + \frac{1}{d_k} = H(d_k)
\end{align}
So we can see from \ref{HarmonicFnProof} that \ref{lemma} is proved. Let $A^*$ be the set of vertex sets formed by the union of all sets with each set consisting of a vertex in the optimal solution along with all the vertices connected to it by edges in the graph G. For the size $C$ of the dominating set found by the algorithm
\begin{align} \label{equation1}
C= \sum \limits_{x \in V} C_x \leq \sum \limits_{S_k \in A^*} \sum \limits_{x \in S_k} C_x
\end{align}
Using \ref{lemma}, \ref{equation1} reduces to
\begin{align*} 
C &\leq \sum \limits_{S_k \in A^*} H(d_k) \leq \sum \limits_{S_k \in A^*} H(d_{max})\\
C &\leq H(d_{max}) \cdot \left | A^* \right |\\
C &\leq O(log(d_{max})) \cdot OPT
\end{align*} 
Since $d_{max}$ (the size of the largest set of a vertex and all vertices connected to it by edges) is limited by the number of vertices $n$ in graph $G$,
\begin{align*} 
C \leq O(log(n)) \cdot OPT\\
\end{align*} 
\begin{align} \label{final}
C = O(OPT \cdot log(n))
\end{align}
Equation \ref{final} proves the requirement.

\item In this exercise, we consider a \enquote{dual} problem of the load balancing problem.

Suppose there are $m$ machines and $n$ jobs such that each job $i$ has a processing time $t_i$.
Consider a job assignment that assigns each job to one of these machines. For each machine
$j$, let $T_j$ denote the total sum of the processing time of all jobs assigned to machine $j$, and
we call $T_j$ the workload of machine $j$. We call the value $min_{1 \leq j \leq m} T_j$ the \textit{minimum workload} of all machines of the assignment.

The \textit{dual load balancing problem} is to compute a job assignment that \textit{maximizes} the minimum
workload of all machines.

\textbf{Remark.} Recall that the load balancing problem is to find a job assignment that minimizes
the maximum workload of all machines. Therefore, the two problem are \enquote{dual} to each other.
\begin{enumerate}
\item \textbf{(5 points)} The dual load balancing problem defined above is an optimization problem. What is the decision version of this problem?

The decision version of the problem is that given $m$ machines and $n$ jobs as described above, is it possible to find an assignment of jobs to machines such that the minimum workload is at least $k$.

\item \textbf{(10 points)} Prove that the decision problem is NP-Complete.

We know that the partition problem is in NP-Complete. As a recap, the partition problem is, given a set $A=\{a_1, a_2,...,a_p\}$, to determine if it is possible to partition $A$ into two subsets $A_1$ and $A_2$ ($A=A_1 \cup A_2$ and $A_1 \cap A_2 = \phi$) such that $\sum A_1 = \sum A_2$. If we can reduce the partition problem to our dual load balancing problem in polynomial time, then we can show that our problem is in NP-Hard. Let $(A)$ be an instance of the partition problem, where $A$ is a set of items as described above and let (m, n, k) be an instance of the dual load balancing problem, where $m$, $n$ and $k$ are as described in (a) above. Consider a special case of the dual load balancing problem where $m=2$, $n=A$ and $k = \frac{1}{2} \sum A = \frac{1}{2} \sum t_i$.
\begin{enumerate}
\item We can see that it is possible to reduce the instance of the partition problem to our special instance of the dual load balancing problem in polynomial time.

\item Consider the case where it is possible to partition the set $A$. In this case, we know that it will be possible to have a minimum workload of $k$. Consider the case where it is possible to have a minimum workload of $k$. The way the problem has been constructed, the workload of each machine will be $k$ and this means that the instance of the partition problem will be true. 

\end{enumerate}

From (i) and (ii) above we can conclude that the dual load balancing problem is in NP-Hard since it can be reduced in polynomial time from a known NP-Complete problem.

Given an instance of a dual load balancing problem $(m, n, k)$ and a certificate (an assignment list of jobs to each machine $m$), we can verify in polynomial time that the certificate is correct. So the dual load balancing problem is in NP.

Since the dual load balancing problem is both in NP-Hard and NP, we can conclude that it is in NP-Complete.

\item \textbf{(15 points)} Let $A$ be the sum of the processing time of all jobs, i.e., $A = \sum\limits_{i=1}^n t_i$. We assume that $t_i \leq \frac{A}{2m}$ for each job $i$ (intuitively, each $t_i$ is not \enquote{too big}). Under this assumption, design a polynomial-time approximation algorithm for the problem with
approximation ratio 2. In other words, if $OPT$ is the minimum workload in an optimal
solution and $C$ is the minimum workload in your solution, then it holds that $C \geq \frac{1}{2} \cdot OPT$.
\end{enumerate}
\end{enumerate}

%----------------------------------------------------------------------------------------

\end{document}